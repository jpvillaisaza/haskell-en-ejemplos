\documentclass[spanish]{beamer}

\usetheme{default}

\usepackage{polyglossia}

\setdefaultlanguage{spanish}

\usepackage{fancyvrb}

\DefineShortVerb{\|}
\DefineVerbatimEnvironment{code}{Verbatim}{frame=lines}

\title{Haskell en ejemplos}
\author{Juan Pedro Villa Isaza}
\institute{Stack Builders}
\date{8 de abril de 2016}

\logo{\includegraphics[scale=0.15]{stackbuilders-logo.png}}

\begin{document}

%%%%%%%%%%%%%%%%%%%%%%%%%%%%%%%%%%%%%%%%%%%%%%%%%%%%%%%%%%%%%%%%%%%%%%

\frame{\titlepage}

%%%%%%%%%%%%%%%%%%%%%%%%%%%%%%%%%%%%%%%%%%%%%%%%%%%%%%%%%%%%%%%%%%%%%%

\begin{frame}[fragile]
  \frametitle{¡Hola, Haskell!}

  \begin{code}
main :: IO ()
main = putStrLn "¡Hola, Haskell!"
  \end{code}
\end{frame}

%%%%%%%%%%%%%%%%%%%%%%%%%%%%%%%%%%%%%%%%%%%%%%%%%%%%%%%%%%%%%%%%%%%%%%

\begin{frame}
  \frametitle{Haskell}

  \begin{figure}
    \includegraphics[scale=0.6]{haskell-logo.png}
  \end{figure}

  \begin{center}
    \url{https://www.haskell.org/}
  \end{center}
\end{frame}

%%%%%%%%%%%%%%%%%%%%%%%%%%%%%%%%%%%%%%%%%%%%%%%%%%%%%%%%%%%%%%%%%%%%%%

\begin{frame}
  \frametitle{Haskell es...}

  \begin{itemize}
  \item Funcional
  \item
    Puro
    \begin{itemize}
    \item
      Inmutabilidad
    \item
      Sin efectos secundarios
    \item
      Transparencia referencial
    \end{itemize}
  \item
    Multipropósito
  \item
    ...
  \end{itemize}
\end{frame}

%%%%%%%%%%%%%%%%%%%%%%%%%%%%%%%%%%%%%%%%%%%%%%%%%%%%%%%%%%%%%%%%%%%%%%

\begin{frame}[fragile]
  \frametitle{Haskell es funcional}
  \framesubtitle{Fibonacci}

  \begin{code}
fibonacci :: Integer -> Integer
fibonacci 0 = 0
fibonacci 1 = 1
fibonacci n = fibonacci (n - 1) + fibonacci (n - 2)
  \end{code}

  \begin{code}
> fibonacci 5
5
  \end{code}
\end{frame}


\begin{frame}[fragile]
  \frametitle{Haskell es funcional}
  \framesubtitle{Fibonacci}

  \begin{code}
int fibonacci(int n) {
  if (n == 0)
    return 0;
  else if (n == 1)
    return 1;
  else
    return fibonacci (n - 1) + fibonacci (n - 2);
}
  \end{code}
\end{frame}

%%%%%%%%%%%%%%%%%%%%%%%%%%%%%%%%%%%%%%%%%%%%%%%%%%%%%%%%%%%%%%%%%%%%%%

\begin{frame}<1>[fragile, label=factorial]
  \frametitle{Haskell es puro}
  \framesubtitle{Factorial}

  \begin{code}
factorial :: Integer -> Integer
factorial 0 = 1
factorial n = n * factorial (n - 1)
  \end{code}

  \begin{code}
> factorial 5
120
  \end{code}
\end{frame}


\begin{frame}[fragile]
  \frametitle{Haskell es puro}
  \framesubtitle{Factorial}

  \begin{code}
int factorial(int n) {
  int factorial = 1;
  while (n > 0) {
    factorial = factorial * n;
    n = n - 1;
  }
  return factorial;
}
  \end{code}
\end{frame}


\begin{frame}[fragile]
  \frametitle{Haskell es puro}
  \framesubtitle{Factorial}

  \begin{code}
int factorial(int n) {
  if (n == 0)
    return 1;
  else
    return n * factorial(n - 1);
}
  \end{code}
\end{frame}


\againframe<2>{factorial}

%%%%%%%%%%%%%%%%%%%%%%%%%%%%%%%%%%%%%%%%%%%%%%%%%%%%%%%%%%%%%%%%%%%%%%

\begin{frame}[fragile]
  \frametitle{Haskell es puro y funcional}
  \framesubtitle{\texttt{reverse}}

  \begin{code}
reverse :: [a] -> [a]
reverse []     = []
reverse (x:xs) = reverse xs ++ [x]
  \end{code}

  \begin{code}
> reverse [0,1,2,3,4,5]
[5,4,3,2,1,0]
  \end{code}
\end{frame}


\begin{frame}[fragile]
  \frametitle{Haskell es puro y funcional}
  \framesubtitle{\texttt{reverse}}

  \begin{code}
propReverse :: [Integer] -> Bool
propReverse xs = reverse (reverse xs) == xs
  \end{code}

  \begin{code}
> quickCheck propReverse
+++ OK, passed 100 tests.
  \end{code}
\end{frame}

%%%%%%%%%%%%%%%%%%%%%%%%%%%%%%%%%%%%%%%%%%%%%%%%%%%%%%%%%%%%%%%%%%%%%%

\begin{frame}
  \frametitle{Haskell tiene...}

  \begin{itemize}
  \item
    Funciones de orden superior
  \item
    Evaluación perezosa
  \item
    ...
  \end{itemize}
\end{frame}

%%%%%%%%%%%%%%%%%%%%%%%%%%%%%%%%%%%%%%%%%%%%%%%%%%%%%%%%%%%%%%%%%%%%%%

\begin{frame}[fragile]
  \frametitle{Haskell tiene funciones de orden superior}
  \framesubtitle{\texttt{map}}

  \begin{code}
map :: (a -> b) -> [a] -> [b]
map _ []     = []
map f (x:xs) = f x : map f xs
  \end{code}

  \begin{code}
> map even [0,1,2,3,4,5]
[True,False,True,False,True,False]
  \end{code}
\end{frame}

%%%%%%%%%%%%%%%%%%%%%%%%%%%%%%%%%%%%%%%%%%%%%%%%%%%%%%%%%%%%%%%%%%%%%%

\begin{frame}[fragile]
  \frametitle{Haskell tiene funciones de orden superior}
  \framesubtitle{\texttt{map}}

  Los diez primeros números de Fibonacci (en C):
  \begin{code}
for (int i = 1; i <= 10; i++) {
  printf("%d\n", fibonacci(i - 1));
}
  \end{code}
\end{frame}


\begin{frame}[fragile]
  \frametitle{Haskell tiene funciones de orden superior}
  \framesubtitle{\texttt{map}}

  Los diez primeros números de Fibonacci (en Haskell):
  \begin{code}
> map fibonacci [0..9]
[0,1,1,2,3,5,8,13,21,34]
  \end{code}
\end{frame}


\begin{frame}[fragile]
  \frametitle{Haskell tiene evaluación perezosa}
  \framesubtitle{\texttt{map}}

  Los diez primeros números de Fibonacci (en Haskell):
  \begin{code}
> let fibonaccis = map fibonacci [0..]
> take 10 fibonaccis
[0,1,1,2,3,5,8,13,21,34]
  \end{code}
\end{frame}

%%%%%%%%%%%%%%%%%%%%%%%%%%%%%%%%%%%%%%%%%%%%%%%%%%%%%%%%%%%%%%%%%%%%%%

\begin{frame}[fragile]
  \frametitle{Haskell tiene funciones de orden superior}
  \framesubtitle{\texttt{filter}}

  \begin{code}
filter :: (a -> Bool) -> [a] -> [a]
filter _ []     = []
filter p (x:xs)
  | p x         = x : filter p xs
  | otherwise   = filter p xs
  \end{code}

  \begin{code}
> filter odd [0..9]
[1,3,5,7,9]
  \end{code}
\end{frame}


\begin{frame}[fragile]
  \frametitle{Haskell tiene funciones de orden superior}
  \framesubtitle{\texttt{filter}}

  Los diez primeros números pares de Fibonacci (en C):
  \begin{code}
for (int i = 1, j = 0; i <= 10; j++) {
  if (fibonacci(j) % 2 == 0) {
    printf("%d\n", fibonacci(j));
    i = i + 1;
  }
}
  \end{code}
\end{frame}


\begin{frame}[fragile]
  \frametitle{Haskell tiene funciones de orden superior}
  \framesubtitle{\texttt{filter}}

  Los diez primeros números pares de Fibonacci (en Haskell):
  \begin{code}
> take 10 (filter even fibonaccis)
[0,2,8,34,144,610,2584,10946,46368,196418]
  \end{code}
\end{frame}

%%%%%%%%%%%%%%%%%%%%%%%%%%%%%%%%%%%%%%%%%%%%%%%%%%%%%%%%%%%%%%%%%%%%%%

\begin{frame}[fragile]
  \frametitle{Haskell tiene funciones de orden superior}
  \framesubtitle{\texttt{foldr}}

  \begin{code}
foldr :: (a -> b -> b) -> b -> [a] -> b
foldr _ n []     = n
foldr c n (x:xs) = c x (foldr c n xs)
  \end{code}

  \begin{code}
sum :: [Integer] -> Integer
sum ns = foldr (+) 0 ns
  \end{code}
\end{frame}


\begin{frame}[fragile]
  \frametitle{Haskell tiene funciones de orden superior}
  \framesubtitle{\texttt{foldr}}

  \begin{code}
foldr :: (a -> b -> b) -> b -> [a] -> b
foldr _ n []     = n
foldr c n (x:xs) = c x (foldr c n xs)
  \end{code}

  \begin{code}
sum :: [Integer] -> Integer
sum = foldr (+) 0
  \end{code}
\end{frame}


\begin{frame}[fragile]
  \frametitle{Haskell tiene funciones de orden superior}
  \framesubtitle{\texttt{foldr}}

  La suma de los diez primeros números pares de Fibonacci (en C):
  \begin{code}
int s = 0;
for (int i = 1, j = 0; i <= 10; j++) {
  if (fibonacci(j) % 2 == 0) {
    s = s + fibonacci(j);
    i = i + 1;
  }
}
printf("%d\n", s);
  \end{code}
\end{frame}


\begin{frame}[fragile]
  \frametitle{Haskell tiene funciones de orden superior}
  \framesubtitle{\texttt{foldr}}

  La suma de los diez primeros números pares de Fibonacci (en
  Haskell):
  \begin{code}
> sum (take 10 (filter even fibonaccis))
257114
  \end{code}
\end{frame}

%%%%%%%%%%%%%%%%%%%%%%%%%%%%%%%%%%%%%%%%%%%%%%%%%%%%%%%%%%%%%%%%%%%%%%

\begin{frame}
  \frametitle{Haskell en ejemplos}

  \begin{center}
    \url{https://github.com/stackbuilders/haskell-en-ejemplos}
  \end{center}
\end{frame}

%%%%%%%%%%%%%%%%%%%%%%%%%%%%%%%%%%%%%%%%%%%%%%%%%%%%%%%%%%%%%%%%%%%%%%

\end{document}
